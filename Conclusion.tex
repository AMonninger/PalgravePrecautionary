%\documentclass[12pt,letterpaper]{article}\usepackage{epsfig,amsmath,amssymb,natbib,ifthen,hyperref}\provideboolean{StandAlone}\setboolean{StandAlone}{true} \begin{document}

The qualitative and quantitative aspects of the theory of
precautionary behavior are now well established. Less agreement exists
about the strength of the precautionary saving motive and the
magnitude of precautionary wealth. Structural models that match broad
features of consumption and saving behavior tend to produce estimates
of the degree of prudence that are less than those obtained from
theoretical models in combination with risk aversion estimates from
survey evidence.  Direct estimates of precautionary wealth seem to be
sensitive to the exact empirical procedures used, and are subject to
problems of unobserved heterogeneity that have been demonstrated from
German data after reunification. Thus, establishing the intensity of
the precautionary saving motive and the magnitude of precautionary
wealth remain lively areas of debate.


\ifthenelse{\boolean{StandAlone}}{\end{document}}{}
