\documentclass[12pt,letterpaper]{econtex}
\usepackage{econtexSetup}\usepackage{econtexShortcuts}
\provideboolean{StandAlone}\setboolean{StandAlone}{false}
\begin{document}

% Updated Feb 2 to include JEL codes


% Fuzz -------------------------------------------------------------------
% \hfuzz2pt % Don't bother to report over-full boxes if over-edge is < 2pt
% Line spacing -----------------------------------------------------------



\begin{titlepage}

  \enlargethispage{3000pt}

  \hfill{\tiny \today, \jobname}



  \vspace{0.2in}

  \centerline{\LARGE Precautionary Saving and Precautionary Wealth}

  \vspace{.2in}

  \normalsize

  \centerline{\Large Christopher D. Carroll$^1$}
  % \centerline{ccarroll@jhu.edu}
  \medskip\medskip
  \centerline{\Large Miles S. Kimball$^2$}
  % \centerline{mkimball@umich.edu}
  \medskip
  \medskip


  \centerline{June 12, 2007}

  \vspace{.2in}

  \centerline{\bf Abstract}
  \normalsize

  Precautionary saving measures the consequences of uncertainty for the
  rate of change (and therefore the level) of wealth.  The qualitative
  aspects of precautionary saving theory are now well established: An
  increase in uncertainty will increase the level of saving, but will
  reduce the marginal propensity to save.  Empirical studies using a
  broad range of methodologies have detected evidence of precautionary
  saving behavior, but researchers have not yet achieved consensus on
  how the wide variety of survey and empirical evidence should be
  integrated with theory.

  \vspace{0.2in}

  \noindent Keywords: Precautionary saving, prudence, consumption function, buffer stock saving, 
  marginal propensity to consume
  \medskip\medskip

  \noindent JEL Codes: C61, D11, E21

  \medskip



  \begin{small}
    \parbox{\textwidth}{
      \begin{center}
        \begin{tabbing}
          \texttt{~Archive:~} \= \= \url{http://econ.jhu.edu/people/ccarroll/Papers/PalgravePrecautionary.zip} \\
          \texttt{~~~~~PDF:~} \> \> \url{http://econ.jhu.edu/people/ccarroll/Papers/PalgravePrecautionary.pdf} \\
          \texttt{~~~~~Web:~} \> \>  \url{http://econ.jhu.edu/people/ccarroll/Papers/PalgravePrecautionary/}\\
          \texttt{~~GitHub:~} \> \> \url{http://github.com/llorracc/PalgravePrecautionary} \\
          \texttt{~~~~~~~~~~} \> \> \textit{(In GitHub repo, see \texttt{/Code} for tools for solving and simulating the model)} \\
        \end{tabbing}
      \end{center}
    }


  % \vspace{0.1in}

  \noindent This is an entry for {\it The New Palgrave Dictionary of Economics, 2nd Ed.}  We would like to thank Luigi Guiso, Arthur Kennickell, Annamaria Lusardi, Jonathan Parker, Luigi Pistaferri,
  and Patrick Toch\`e for valuable comments on an earlier draft which resulted in substantial improvements.


  \begin{itemize}
  \item[$^1$] Department of Economics, Johns Hopkins University, Baltimore MD (\texttt{ccarroll@jhu.edu}),
    \\ \url{http://econ.jhu.edu/people/ccarroll}

  \item[$^2$] Department of Economics, University of Michigan, Ann Arbor, MI (\texttt{mkimball@umich.edu}), \url{http://www.umich.edu/~mkimball}
  \end{itemize}

  \end{small}          

\end{titlepage}


\setlength{\footskip}{36pt}

\pagestyle{plain}\thispagestyle{empty}

\titlepage

\tableofcontents

\pagebreak

\section{Introduction}\label{sec:Intro}

\input Intro.tex

\section{Strength of the Precautionary Saving Motive}\label{sec:Strength}

\input Strength.tex

\section{Buffer Stock Wealth}\label{sec:BufferStockWealth}

\input BufferStockWealth.tex

\section{Empirical Evidence}\label{sec:EmpiricalEvidence}

\input EmpiricalEvidence.tex

\section{Conclusion}\label{sec:Conclusion}

\input Conclusion.tex

\pagebreak
\bibliographystyle{econtex}
\input ./bibFileDefine.txt

\end{document}
